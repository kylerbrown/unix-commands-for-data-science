% cheatsheet template from https://tex.stackexchange.com/questions/8827/preparing-cheat-sheets
\documentclass[10pt]{article}
%\documentclass[10pt,landscape]{article}
\usepackage{multicol}
\usepackage{calc}
\usepackage{ifthen}
\usepackage{textcomp}
\usepackage[]{geometry}
%\usepackage[landscape]{geometry}
\usepackage{amsmath,amsthm,amsfonts,amssymb}
\usepackage{color,graphicx,overpic}
\usepackage[pdfborder={0 0 0 [0]}]{hyperref}
\pdfinfo{
  /Title (unix_commands_for_data_science.pdf)
  /Creator (TeX)
  /Producer (pdfTeX 1.40.0)
  /Author (Kyler Brown)
  /Subject (shell)
  /Keywords (unix, shell, data, science, cheatsheet)}
% This sets page margins to .5 inch if using letter paper, and to 1cm
% if using A4 paper. (This probably isn't strictly necessary.)
% If using another size paper, use default 1cm margins.
\ifthenelse{\lengthtest { \paperwidth = 11in}}
    { \geometry{top=.5in,left=.5in,right=.5in,bottom=.5in} }
    {\ifthenelse{ \lengthtest{ \paperwidth = 297mm}}
        {\geometry{top=1cm,left=1cm,right=1cm,bottom=1cm} }
        {\geometry{top=1cm,left=1cm,right=1cm,bottom=1cm} }
    }
% Turn off header and footer
\pagestyle{empty}
% Redefine section commands to use less space
\makeatletter
\renewcommand{\section}{\@startsection{section}{1}{0mm}%
                                {-1ex plus -.5ex minus -.2ex}%
                                {0.5ex plus .2ex}%x
                                {\normalfont\large\bfseries}}
\renewcommand{\subsection}{\@startsection{subsection}{2}{0mm}%
                                {-1explus -.5ex minus -.2ex}%
                                {0.5ex plus .2ex}%
                                {\normalfont\normalsize\bfseries}}
\renewcommand{\subsubsection}{\@startsection{subsubsection}{3}{0mm}%
                                {-1ex plus -.5ex minus -.2ex}%
                                {1ex plus .2ex}%
                                {\normalfont\small\bfseries}}
\makeatother
% Define BibTeX command
\def\BibTeX{{\rm B\kern-.05em{\sc i\kern-.025em b}\kern-.08em
    T\kern-.1667em\lower.7ex\hbox{E}\kern-.125emX}}
% Don't print section numbers
\setcounter{secnumdepth}{0}
\setlength{\parindent}{0pt}
\setlength{\parskip}{0pt plus 0.5ex}
%My Environments
\newtheorem{example}[section]{Example}
\usepackage[T1]{fontenc}
\newcommand{\q}{\textquotesingle } %paste-able single quote
\newcommand{\qq}{\char34 } %paste-able single quote
\newcommand{\tild}{\textasciitilde \ }
% -----------------------------------------------------------------------

\begin{document}
\raggedright
\footnotesize
\begin{multicols}{2}


% multicol parameters
% These lengths are set only within the two main columns
%\setlength{\columnseprule}{0.25pt}
\setlength{\premulticols}{1pt}
\setlength{\postmulticols}{1pt}
\setlength{\multicolsep}{1pt}
\setlength{\columnsep}{2pt}

\begin{center}
     \Large{\underline{Unix commands for data science}} \\
\end{center}

\section{Manipulating input (Bash)}
{\tt ctrl+f} \hfill cursor {\bf f}orward one character\\
{\tt ctrl+b} \hfill cursor {\bf b}ackward one character\\
{\tt alt+f} \hfill cursor {\bf f}orward one word\\
{\tt alt+b} \hfill cursor {\bf b}ackward one word\\
{\tt ctrl+a} \hfill cursor to beginning of line\\
{\tt ctrl+e} \hfill cursor to {\bf e}nd of line\\
{\tt ctrl+p} \hfill {\bf p}revious input\\
{\tt ctrl+n} \hfill {\bf n}ext input\\
{\tt ctrl+r} \hfill {\bf r}everse search previous commands\\
{\tt ctrl+c} \hfill {\bf c}lose running program\\
{\tt ctrl+d} \hfill close shell\\
{\tt ctrl+u} \hfill cut from cursor to beginning of line\\
{\tt ctrl+k} \hfill cut from cursor to end of line ({\bf k}ill)\\
{\tt ctrl+y} \hfill paste ({\bf y}ank)\\
{\tt TAB} \hfill autocomplete\\
{\tt !!} \hfill run previous command\\
{\tt alias l=\qq ls -ltrh\qq} \hfill rename common commands\\
\section{Unix Basics}
\subsection{Getting help}
{\tt ls --help} \hfill prints command {\bf h}elp\\
{\tt man rm} \hfill opens {\bf man}ual for command\\
\subsection{Navigation}
{\tt  ls} \hfill {\bf l}i{\bf s}t directory\\
{\tt pwd} \hfill {\bf p}rint {\bf w}orking {\bf d}irectory\\
{\tt mkdir child} \hfill  {\bf m}a{\bf k}e {\bf dir}ectory named ``child''\\
{\tt cd child} \hfill {\bf c}hange {\bf d}irectory to ``child''\\
{\tt cd ..} \hfill {\bf c}hange {\bf d}irectory to parent\\
{\tt cd} \hfill {\bf c}hange {\bf d}irectory to home (\textasciitilde).\\
{\tt cp file newfile} \hfill {\bf c}o{\bf p}y file\\
{\tt rm file} \hfill {\bf r}e{\bf m}ove (delete) ``file''\\
{\tt rmdir child} \hfill {\bf r}e{\bf m}ove empty {\bf dir}ectory ``child''\\
{\tt find \tild  -name \qq*.csv\qq} \hfill search for files ending with ``*.csv''\\

\subsection{Remote}
{\tt ssh user@example.com} \hfill remote login ({\bf s}ecure {\bf sh}ell)\\
{\tt rsync user@example.com:file local/dir/} \hfill copy remote file\\
{\tt wget http://example.com/data.csv} \hfill copy file from web\\
{\tt curl -O http://example.com/data.csv} \hfill copy file from web\\

\subsection{Managing Processes}
{\tt parallel wc -w ::: *.csv} \hfill process files in parallel\\
{\tt top} \hfill display processes\\
{\tt ps -u} \hfill display user processes\\
{\tt kill 1234} \hfill kill process 1234\\
{\tt crontab -e} \hfill edit cron jobs (run a script daily/weekly/etc.)\\
{\tt sleep 100 \&} \hfill append ``\&'' to run in background\\
{\tt ctrl+z} \hfill stop foreground process\\
{\tt bg} \hfill resume stopped process in {\bf b}ack{\bf g}round\\
{\tt jobs} \hfill list running processes\\
{\tt fg 1} \hfill bring job 1 to {\bf f}ore{\bf g}round\\

\subsection{piping ({\tt <, |, >, >{}>})}
{\tt <{}<{}<} \hfill pass string as input to command\\
{\tt <} \hfill use file as input to command \\
%{\tt tr 'A-Z' 'a-z' < data.csv} \hfill convert letters from uppercase to lowercase\\
{\tt |} \hfill pass output as input\\
%{\tt wc -l * | sort -n | head -5} \hfill 5 smallest files\\
{\tt > } \hfill pass output to file \\
%{\tt grep -v NaN" data.csv > noNaN.csv} \hfill new file without lines containing "NaN"\\
{\tt >{}> } \hfill append output to file \\
%{\tt echo "date, price, quantity" > data2.csv; \\cat data.csv >> data2.csv} \hfill create a new csv with a header line.\\

\subsection{globbing}
{\tt ls *.csv} \hfill list files ending with ``.csv''\\
{\tt ls d*.csv} \hfill list files starting with ``d'' and ending with ``.csv''\\
{\tt ls data????} \hfill list files starting with ``data.'' followed by any 4 \\
	\hfill characters\\
{\tt rm [a-z]*[0-9]} \hfill remove files starting with a letter and ending\\
	\hfill in a digit \\

\section{Data Manipulation}
(assumes data are in comma separated fields)
\subsection{Taking Subsets}
{\tt cat data.csv} \hfill returns contents of ``data.csv''\\
{\tt head data.csv} \hfill first ten lines\\
{\tt tail -15 data.csv} \hfill last 15 lines\\
{\tt tail -n +2 data.csv} \hfill everything but first line (remove header)\\
{\tt cut -d,  -f2 data.csv} \hfill second column\\
{\tt awk -F, \q \{print \$2\}\q\ data.csv} \hfill second column\\
{\tt cut -d,  -f2,4 data.csv} \hfill second and fourth column\\
{\tt cut -d,  -f2 -{}-complement data.csv} \hfill everything except second column\\
{\tt grep \qq NaN\qq\  data.csv} \hfill all lines with a ``NaN''\\
{\tt grep -v \qq NaN\qq\  data.csv} \hfill all lines \emph{without} a ``NaN''\\
{\tt sort data.csv | uniq } \hfill only {\bf uniq}ue lines\\
{\tt sort data.csv | uniq -d } \hfill only {\bf d}uplicate lines\\
{\tt shuf data.csv} \hfill {\bf shuf}fle lines\\
{\tt shuf data.csv | head -1} \hfill random line\\

\subsection{Transforming Data}
{\tt nano data.csv} \hfill minimal text editor\\
{\tt sort data.csv} \hfill sort lines alphabetically\\
{\tt sort -t, -n -k 2 data.csv} \hfill sort lines numerically by column 2\\
{\tt sed \q /s/,/ /g\q \ data.csv} \hfill replace string ``,'' with a space\\
{\tt tr \q A-Z\q\ \q a-z\q \  < data.csv} \hfill convert letters to lowercase\\
{\tt awk -F, \q\{print \$1/100\}\q\ data.csv} \hfill divide column 1 by 100\\
{\tt awk -F, \q\{print \$1*\$2\}\q\ data.csv} \hfill multiply columns 1 and 2\\
{\tt paste -s -d, data.csv } \hfill flatten data to row\\
{\tt tr \q,\q\ \q \textbackslash n\q\ < data.csv} \hfill flatten data to column\\
{\tt paste -d, data1.csv data2.csv} \hfill combines the lines of two files\\
{\tt join -d, data1.csv data2.csv} \hfill performs a join of two files\\

\subsection{Summarizing Data}
{\tt sed \q s/,/ /g\q\  data.csv | wc -w} \hfill {\bf w}ord {\bf c}ount\\
{\tt wc -l data.csv} \hfill number of {\bf l}ines in ``data.csv''\\
{\tt grep -c \qq NaN\qq \  data.csv} \hfill number of lines with a ``NaN''\\
{\tt grep -o \qq NaN\qq \  data.csv | wc -l} \hfill total number of ``NaN''\\
{\tt awk -F, \q \{sum += \$1\} END \{print sum\}\q\  data.csv} \hfill sum of column 1\\
{\tt awk -F, \q \{sum += \$3\} END \{print sum / NR\}\q\  data.csv } \hfill average of\\
	\hfill column 3\\
{\tt awk -F, \q \{sum+=\$2; sumsq+=\$2*\$2\}END\{print sqrt(sumsq/NR- (sum/NR)**2)\}\q \  data.csv} \hfill standard deviation of column 2\\
{\tt cut -d, -f2 data.csv | sort -n | head -1} \hfill minimum of column 2\\
{\tt cut -d, -f2 data.csv | sort -n | tail -1} \hfill maximum of column 2\\
{\tt tr \q ,\q\ \q\textbackslash n\q\ < data.csv | sort -n | tail -1} \hfill maximum of all columns\\

\subsection{Generating Data}
{\tt echo \$((123 * 456))} \hfill integer calculator\\
{\tt bc <{}<{}< \qq 12.3 * 456\qq} \hfill calculator\\
{\tt seq 3 11} \hfill {\bf seq}uence of numbers, inclusive\\
{\tt shuf -r -i 0-100 -n 10} \hfill 10 random numbers between 0 and 100 with\\
	\hfill replacement\\
{\tt shuf -i 0-100 -n 10} \hfill 10 random numbers between 0 and 100 \emph{without}\\ 
	\hfill replacement\\

\section{Rapid Visualization with \href{https://github.com/dkogan/feedgnuplot}{feedgnuplot}}
{\tt sudo apt-get install feedgnuplot}\\
{\tt cut -d, -f2  data.csv | feedgnuplot --terminal \q dumb\q} \hfill plot column 2\\
	\hfill in terminal\\
{\tt  cut -d, -f2  data.csv | feedgnuplot -{}-line} \hfill line plot column 2\\
{\tt awk \q\{print \$3, \$2\}\q\  data.csv | feedgnuplot -{}-domain} \hfill plot column\\
	\hfill 2 over column 3\\
{\tt cut -d, -f2   data.csv | feedgnuplot -{}-histogram -{}-with boxes}\\
	\hfill histogram of column 2\\

% You can even have references
\rule{0.3\linewidth}{0.25pt}\\
\scriptsize
\href{https://creativecommons.org/licenses/by/4.0/}{CC BY 4.0} Kyler Brown\\
source: \url{https://github.com/kylerbrown/unix-commands-for-data-science}\\

\textit{further reading}\\
\href{https://www.gnu.org/software/coreutils/manual/coreutils.html}{{\tt info coreutils}}\\
\url{http://datascienceatthecommandline.com/}\\
\url{http://www.drbunsen.org/explorations-in-unix/}\\
\url{http://www.gregreda.com/2013/07/15/unix-commands-for-data-science/}\\
\bibliographystyle{abstract}
\bibliography{refFile}
\end{multicols}
\end{document}